\documentclass[11pt,a4paper]{article}
\usepackage[utf8]{inputenc} 
\usepackage{graphicx} 
\usepackage[left=2cm,top=2cm,right=2cm,bottom=1.5cm]{geometry}
\usepackage{amsmath} %for mathThe WS\_2017 Game \\
\usepackage{fancyhdr}
\usepackage{listings} %for code
\usepackage{hyperref}

\pagestyle{fancy}


\lhead{The WS\_2017 Game \\ Programm Architecture}
\rhead{Flavia Brogle \\
	 Max Hackinger}
\cfoot{\thepage}

\title{The WS\_2017 Game \\ The Programm Architecture}
\author{Flavia Brogle \\ Max Hackinger}

\renewcommand{\headrulewidth}{0.4pt}
\renewcommand{\footrulewidth}{0.4pt}
\begin{document}

	\maketitle
  \tableofcontents
  
\begin{abstract}
  This article describes the Program Architecture of the WS\_2017 Game, starting with a quick overview of all the pieces of software(components ?) involved. In the next three sections it goes in to grater detail about how the Client and Server side are constructed and how they communicate. There are some schematics that try to create a visual overview of some of the (components ?) 
\end{abstract}  
  \clearpage
	\section{Quick over view of the Game Program}
	  WS\_2017 is a round based game that consists of a clients and a server which communicate through the WS\_2017 protocol. The protocol is inspired by the designe of the POP3 protocol and uses commands that consist of a keyword and arguments.
		\subsection{}
		\subsection{}
		\subsection{}
	\clearpage
	\section{Client side}
		\subsection{}
		\subsection{}
		\subsection{} 
	\clearpage
	\section{Server side}
		\subsection{Sequence of Events}
		When a server instance is first spun up, a main thread is crated that in turn spins up a new thread for every incoming TCP connection. These individual threads handle the processing and validation of the protocol between their clients. All the Structures that Need to be accessed by all clients are in the main thread, this includes... . 
		\subsection{}
		\subsection{}
\end{document}
